% !TeX spellcheck = ru_RU
% !TEX root = vkr.tex

\section{Введение}
\thispagestyle{withCompileDate}

Местом действия игры Time Reactor\footnote{https://github.com/RuslanBeresnev/Time-Reactor-Game (дата обращения: \DTMdate{2023-09-27}).} является бесконечная лестница, на этажах которой случайным образом располагаются помещения одного из трёх типов: лаборатория, компьютерный центр и склад технического оборудования.
Поскольку число комнат, доступных для изучения, не ограничено, то ручное заполнение помещений может вызвать ряд трудностей при создании сотен и тысяч различных интерьерных решений. Кроме того, чем дальше игрок будет продвигаться по лестнице, тем больше уровни будут казаться ему однотипными и похожими друг на друга. Создание большого количества разнообразных интерьеров возможно с помощью методов \textit{процедурной генерации}.

Под процедурной генерацией контента (Procedural Content Ge\-ne\-ra\-ti\-on, \textit{PCG}) понимается наполнение игрового мира с помощью алгоритмов при прямом или косвенном участии пользователя \cite{article}. Методы PCG широко используются при проектировании и разработке игр \cite{dahren2021usage} --- от генерации целых уровней в Spelunky (Mossmouth 2009) и планет
в No Man’s Sky (Hello Games 2016) до характеров персонажей и сюжетов в RimWorld (Ludeon Studios 2013). 

В книге \enquote{Procedural generation in game design} Short Tanya и Adams Tarn приводят ряд причин, побуждающих разработчиков использовать процедурную генерацию в своих проектах. Исходя из них, можно выделить основные мотивы для применения методов PCG в игре Time Reactor для генерации интерьера комнат:
\begin{itemize}
    \item создание разнообразного контента в масштабах, недостижимых вручную;
    \item отсутствует необходимость в хранении сгенерированных данных;
    \item использование PCG может занять сравнительно меньше времени, чем ручное создание того же объёма данных;
    \item повторное использование генераторов.
\end{itemize}

В свою очередь, \textit{процедурная генерация помещений} направлена на создание виртуальной среды, приближенной к реальной, с помощью алгоритмов PCG. Процедурно генерируемое помещение строится из описания объектов, характерных для данного пространства, и связей между ними. К последнему можно отнести иерархические паттерны (посуда на кухне располагается строго на полках, стулья в кабинете за письменным столом) и функциональные требования (экран телевизора находится в зоне видимости и не заставлен другими предметами). 

Для формирования реалистичного игрового мира, система генерации интерьеров должна отвечать и другим требованиям. В частности, активировать алгоритм процедурной генерации следует в наиболее подходящий момент, например, при попадании комнаты в зону видимости игрока (в случае Time Reactor --- во время инициализации всей комнаты). Сам процесс генерации обязан происходить достаточно быстро, чтобы избежать ситуаций, когда игрок заходит в комнату ещё до окончания расстановки мебели. Кроме того, интерьер должен оставаться неизменным при каждом следующем посещении.

Таким образом, необходимо не только реализовать алгоритм для процедурной генерации реалистичного интерьера комнат, но и обеспечить его эффективную интеграцию с остальными игровыми процессами, а также неизменность результата при многократном вызове.