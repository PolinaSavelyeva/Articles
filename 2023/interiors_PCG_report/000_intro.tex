% !TeX spellcheck = ru_RU
% !TEX root = vkr.tex

\section{Введение}
\thispagestyle{withCompileDate}

Time Reactor\footnote{https://github.com/RuslanBeresnev/Time-Reactor-Game (дата обращения: \DTMdate{2023-12-16}).} --- это научно-фантастическая 3D игра на \textit{Unity}, разработанная студентом СПбГУ Русланом Бересневым в качестве семестровой практики. К задачам по расширению игрового мира Time Reactor относится реализация \textit{процедурной генерации} интерьера комнат.  Выполнение этой задач предоставляет возможность не только изучить основы Unity Engine и процедурной генерации, но и добавить новые игровые механики в перспективный проект.

Основным местом действия игры является \enquote{бесконечная} лестница. На ее этажах случайным образом располагаются помещения одного из трех типов: лаборатория, компьютерный центр и склад. Игроку предстоит посетить около сотни подобных помещений прежде, чем он доберется до финальной комнаты. Ручное заполнение подобных локаций представляется объемной задачей с существенными временными затратами \cite{short2017procedural}. В качестве альтернативного подхода по созданию большого количества разнообразных интерьеров используются методы процедурной генерации.
    
Под процедурной генерацией контента (Procedural Content Ge\-ne\-ra\-ti\-on, \textit{PCG}) понимается наполнение игрового мира с помощью алгоритмов \cite{article}. Методы PCG широко используются при разработке игр \cite{dahren2021usage} --- от генерации целых уровней в Spelunky (Mossmouth 2009) и планет
в No Man’s Sky (Hello Games 2016) до характеров персонажей и сюжетов в RimWorld (Ludeon Studios 2013). В первой главе книги \enquote{Procedural generation in game design} Tanya Short и Tarn Adams перечисляют ряд причин, побуждающих разработчиков использовать методы процедурной генерации в своих проектах:
\begin{itemize}
    \item создание разнообразного контента в масштабах, недостижимых вручную;
    \item отсутствует необходимость в хранении сгенерированных данных;
    \item грамотно спроектированный алгоритм позволит сэкономить время на ручном заполнении комнат;
    \item повторное использование генераторов.
\end{itemize}

В свою очередь, \textit{процедурная генерация помещений} направлена на создание виртуальной среды, приближенной к реальной, с помощью алгоритмов PCG. Процедурно генерируемое помещение строится из описания объектов, характерных для данного пространства, и связей между ними. К последнему можно отнести \textit{иерархические паттерны} (посуда на кухне располагается строго на полках, стулья в кабинете за письменным столом) и функциональные требования (экран телевизора находится в зоне видимости и не заставлен другими предметами).

Для придания большей реалистичности игровому миру система генерации должна отвечать и другим требованиям. В частности, активировать  алгоритм следует в наиболее подходящий момент, например, при попадании комнаты в зону видимости. Процесс расстановки следует проводить достаточно быстро, чтобы избежать ситуаций, когда мебель появляется на глазах у игрока. Также немаловажно сохранить интерьер неизменным при каждом следующем посещении.

Таким образом, необходимо не только реализовать алгоритм для процедурной генерации реалистичного интерьера комнат, но и обеспечить его эффективную интеграцию с остальными игровыми процессами, а также неизменность результата при многократном вызове.