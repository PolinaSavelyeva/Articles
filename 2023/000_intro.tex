% !TeX spellcheck = ru_RU
% !TEX root = vkr.tex

\section*{Введение}
\thispagestyle{withCompileDate}

\textit{Графы}, представляющие набор объектов и их отношений, используются для моделирования сложных систем в различных сферах жизни от математики и биологии до социологии и лингвистики. 
При работе с графовыми моделями анализ и эффективное хранение больших объёмов данных являются серьёзными проблемами. Кроме того, во многих прикладных задачах необходимо оперировать \textit{разреженными} структурами. 

Приведём простой пример: рассмотрим граф социальной сети, пользователи которого представлены в виде вершин, а дружеские отношения между ними --- в виде рёбер. В такой модели каждый пользователь имеет небольшую часть из всех возможных связей. Следовательно, матрицу смежности искомого графа можно назвать разреженной --- значительная часть её ячеек будет заполнена нулями. 

Эффективность алгоритма, работающего с разреженными данными, существенно зависит от способа их хранения. Так, в рассмотренном примере представление матрицы смежности графа в виде двумерного массива будет крайне неэффективным не только с точки зрения затрат памяти, но и поиска информации.

Процесс поиска информации в графе главным образом представляет собой обход всех вершин и рёбер. Два наиболее известных алгоритма поиска --- это обход в ширину \textit{Breadth-First Search} (BFS) и обход в длину Depth-First Search (DFS). В данной работе алгоритм DFS будет упомянут лишь косвенно, но нельзя не упомянуть его основное отличие от BFS --- обход в ширину последовательно рассматривает смежные вершины, а обход в длину --- вершины, расположенные по одному пути. Благодаря этому различию в BFS возможно использование матрично-векторных операций (SpMSpV) для эффективной работы с разреженными данными \cite{LinearAlgebra-basedGraphFramework}.

Вследствие разделения данных на части и обработки фрагментов в разных потоках, операции над векторами и матрицами могут быть эффективно \textit{распараллелены}. Это особенно полезно в масштабных вычислениях, где параллельная реализация может значительно ускорить общее время работы.

\textit{Экспериментальное исследование} является важным аспектом при разработке эффективного алгоритма. Во-первых, экспериментальный анализ позволяет оценить производительность метода на реальных данных и его пригодность для практического применения. Во-вторых, способствует пониманию того, как различные аппаратные конфигурации влияют на производительность. И вместе с тем результаты исследования помогают при оптимизации конкретных сценариев работы алгоритма, в которых входные данные могут сильно различаться по размеру и плотности. 