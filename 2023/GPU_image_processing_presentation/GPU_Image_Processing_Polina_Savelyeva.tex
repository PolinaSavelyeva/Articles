%\documentclass{beamer}
%Для защит онлайн лучше использовать разрешение 16x9
\documentclass[aspectratio=169]{beamer}

\input{preamble.tex}

% То, что в квадратных скобках, отображается внизу по центру каждого слайда. 
\title[GPU Image Processing]{Реализация библиотеки по обработке изображений с использованием GPU для платформы .NET}

% То, что в квадратных скобках, отображается в левом нижнем углу. 
\institute[СПбГУ]{}

% То, что в квадратных скобках, отображается в левом нижнем углу.
\author[Савельева Полина]{Савельева Полина Андреевна, группа 22.Б07-мм}
 
\begin{document}
{
\setbeamertemplate{footline}{}
% Лого университета или организации, отображается в шапке титульного листа
\begin{frame}
  \includegraphics[width=1.4cm]{pictures/SPbGU_Logo.png}
\vspace{-35pt}
\hspace{-10pt}
\begin{center}
   \begin{tabular}{c}
        \scriptsize{Санкт-Петербургский государственный университет} \\
        %\scriptsize{Кафедра системного программирования}
    \end{tabular}
\titlepage
\end{center}

\btVFill

{\scriptsize
  % У научного руководителя должна быть указана научная степень
   \textbf{Научный руководитель:} к.~ф.-м.~н. С.~В.~Григорьев, доцент кафедры информатики \\
  % Консультанта может и не быть. Должна быть указана должность или ученая степень
   %\textbf{Консультант:}  П.П. Петров, программист ЗАО \enquote{Компания с ну очень-очень-очень длинным названием}\\
  % Для курсовой не обязателен. Должна быть указана должность или ученая степень
   %\textbf{Рецензент:} д.т.н., проф. И.И. Иванов, исполнительный директор ООО \enquote{Рога и копыта}  
 }
\begin{center}
  \vspace{5pt}
  \scriptsize{Санкт-Петербург\\
                 2023}
  \end{center}

\end{frame}
}

\begin{frame}[fragile]  
  \frametitle{Введение}
  \begin{itemize}
    \item Использование GPU значительно ускоряет обработку изображений и снижает нагрузку на CPU
    \item Доступность графических процессоров позволяет применять их в сферах, где требуется обрабатывать большое количество данных (компьютерное зрение, медицина, графика и дизайн)
    \item Идея --- предоставить .NET разработчикам инструмент, позволяющий эффективно использовать возможности GPU при работе с изображениями
  \end{itemize}
\end{frame}
            
\begin{frame}  
  \frametitle{Существующие решения}
  
  \begin{itemize}
    \item
    \item Указать их преимущества и недостатки (критика существующих решений/подходов)  
     Возможно, предметная область сложна и потребуется больше одного слайда, но затягивать введение не стоит. Постарайтесь уложиться в 1--2 слайда
  \end{itemize}
  \begin{itemize}
    \item Выводы
    \begin{itemize}
      \item Подвести итог
      \item Указать недостатки существующих подходов, на борьбу с которыми 
направленна данная работа
      \item Чётко сформулировать существующую проблему, которая будет решаться в данной работе
  \end{itemize}
  \end{itemize}
    
\end{frame}

% Обязательный слайд: четкая формулировка цели данной работы и постановка задачи
% Описание выносимых на защиту результатов, процесса или особенностей их достижения и т.д.
\begin{frame}
  \frametitle{Постановка задачи}
  \textbf{Целью} работы является реализация библиотеки по обработке изображений с использованием GPU %озвученной выше 
  
  \textbf{Задачи}:
  \begin{itemize}
    \item Добавить следующие возможности:
    \begin{itemize}
        \item Сохранение в различных форматах (исходное изображение в png сохранить в jpeg)
        \item Скос (skew)
        \item Изменение размера (reseize)
        \item Обрезка (crop)
        \item Добавление водяного знака (watermark)
    \end{itemize}
    \item Сравнить производительности текущей реализации и аналогов 
    \item Оформить и опубликовать GitHub пакет
  \end{itemize}
\end{frame}
            
%Идеально, если есть по одному слайду на каждую поставленную задачу            
\begin{frame}
  \frametitle{Языки .NET}
  \begin{minipage}[m]{0.6\linewidth}
        \begin{figure}
            \centering
            \includegraphics[width=8.0cm]{pictures/CIL.pdf}
            \caption{Трансляция языков .NET}
            \label{fig:cil}
        \end{figure}
    \end{minipage}\hfill
    \begin{minipage}[m]{0.4\linewidth}
        Выбор .NET языков при написании библиотеки может иметь несколько причин:
    \begin{itemize}
        \item Богатая экосистема .NET
        \item Переносимость
        \item Высокая производительность
    \end{itemize}
    \end{minipage}
\end{frame}

\begin{frame}
  \frametitle{GPU-вычисления}
  \begin{minipage}[m]{0.6\linewidth}
        \begin{figure}
            \centering
            \includegraphics[width=8.0cm]{pictures/GPU_vs_CPU.pdf}
            \caption{Устройство CPU и GPU}
            \label{fig:cpuvsgpu}
        \end{figure}
    \end{minipage}\hfill
    \begin{minipage}[m]{0.4\linewidth}
        \begin{itemize}
            \item Если в CPU могут быть 2--16 ядер, то в GPU их сотни и тысячи
            \item Наличие нескольких ядер обеспечивает параллелизм и высокую эффективность обработки изображений
            \item OpenCL и Brahma.FSharp
        \end{itemize}
    \end{minipage}
\end{frame}

\end{document}