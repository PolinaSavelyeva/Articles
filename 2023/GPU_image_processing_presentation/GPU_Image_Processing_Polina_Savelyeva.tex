%\documentclass{beamer}
%Для защит онлайн лучше использовать разрешение 16x9
\documentclass[aspectratio=169]{beamer}

\input{preamble.tex}

% То, что в квадратных скобках, отображается внизу по центру каждого слайда. 
\title[GPU Image Processing]{Реализация библиотеки по обработке изображений с использованием GPU для платформы .NET}

% То, что в квадратных скобках, отображается в левом нижнем углу. 
\institute[СПбГУ]{}

% То, что в квадратных скобках, отображается в левом нижнем углу.
\author[Савельева Полина]{Савельева Полина Андреевна, группа 22.Б07-мм}
 
\begin{document}
{
\setbeamertemplate{footline}{}
% Лого университета или организации, отображается в шапке титульного листа
\begin{frame}
  \includegraphics[width=1.4cm]{pictures/SPbGU_Logo.png}
\vspace{-35pt}
\hspace{-10pt}
\begin{center}
   \begin{tabular}{c}
        \scriptsize{Санкт-Петербургский государственный университет} \\
        %\scriptsize{Кафедра системного программирования}
    \end{tabular}
\titlepage
\end{center}

\btVFill

{\scriptsize
  % У научного руководителя должна быть указана научная степень
   \textbf{Научный руководитель:} к.~ф.-м.~н. С.~В.~Григорьев, доцент кафедры информатики \\
  % Консультанта может и не быть. Должна быть указана должность или ученая степень
   %\textbf{Консультант:}  П.П. Петров, программист ЗАО \enquote{Компания с ну очень-очень-очень длинным названием}\\
  % Для курсовой не обязателен. Должна быть указана должность или ученая степень
   %\textbf{Рецензент:} д.т.н., проф. И.И. Иванов, исполнительный директор ООО \enquote{Рога и копыта}  
 }
\begin{center}
  \vspace{5pt}
  \scriptsize{Санкт-Петербург\\
                 2023}
  \end{center}

\end{frame}
}

\begin{frame}[fragile]  
  \frametitle{Введение}
  \begin{itemize}
    \item Обработка изображений используется в компьютерном зрении, медицине, графике
    \item Использование GPU может значительно ускорить обработку изображений
    \item Популярная платформа .NET предоставляет инструменты для высокоуровневого программирования, в том числе для обработки изображений 
    \item Необходимо средство, позволяющее эффективно использовать возможности GPU при работе с изображениями на платформе .NET
  \end{itemize}
\end{frame}
            
\begin{frame}  
  \frametitle{Существующие решения}
    \begin{itemize}
    
        \item \textbf{Magick.NET}: .NET--обертка для \textbf{ImageMagick}, позволяющая использовать функциональность ImageMagick. Поддерживает множество операций над изображениями, но только на CPU (в отличии от ImageMagick).
    
        \item \textbf{ImageSharp}: помимо основных инструментов предоставляет своего рода строительные блоки, с помощью которых можно разрабатывать дополнительные функции. Не имеет встроенных функций для работы с GPU.

        \item .NET библиотеки для обработки на GPU в большинстве являются обёртками, которые сложно сопровождать, или ориентированы на отдельных производителей видеокарт
    \end{itemize}
    
\end{frame}

% Обязательный слайд: четкая формулировка цели данной работы и постановка задачи
% Описание выносимых на защиту результатов, процесса или особенностей их достижения и т.д.
\begin{frame}
  \frametitle{Постановка задачи}
  \textbf{Целью} работы является реализация библиотеки по обработке изображений с использованием GPU на платформе .NET %озвученной выше 
  
  \textbf{Задачи}:
  \begin{itemize}
    \item Реализовать следующие возможности:
    \begin{itemize}
        \item Сохранение в различных форматах
        \item Изменение размера (reseize)
        \item Обрезка (crop)
        \item Добавление водяного знака (watermark)
    \end{itemize}
    \item Сравнить производительности текущего решения и аналогов 
    \item Оформить и опубликовать nuget пакет
  \end{itemize}
\end{frame}

\begin{frame}
  \frametitle{Brahma.FSharp}
    \begin{itemize}
        \item Переносимость, вследствие использования OpenCL для взаимодействия с GPU 
        \item Высокоуровневость --- весь код на F\#, даже исполняемый на GPU
        \item Дополнительные преимущества F\#
        \begin{itemize}
            \item Примитивы для асинхронного программирования
            \item Система типов
        \end{itemize}
    \end{itemize}
\end{frame}

\end{document}
