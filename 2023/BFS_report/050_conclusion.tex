% !TeX spellcheck = ru_RU
% !TEX root = vkr.tex

\section*{Заключение}

В рамках выполнения данной работы были получены следующие результаты.

\begin{itemize}
\item Реализованы тип-матрица и тип-вектор с использованием деревьев квадрантов в качестве метода хранения разреженных структур, а также векторно-матричные операции над ними.
\item Реализован алгоритм обхода графа в ширину с использованием линейной алгебры.
\item Реализованы параллельные версии векторно-матричных операций с помощью выражений \texttt{async} --- специальной конструкции языка F\#.
\item Выполнено экспериментальное исследование реализованного алгоритма. Обход в ширину с использованием параллельной версии матрично-векторных операций показал ускорение до 3.45 раз на матрицах среднего размера в сравнении с однопоточной версией и не показал себя эффективно на матрицах малого и большого размеров.
\end{itemize}

Результаты исследования могут иметь практическое применение в областях, где требуется анализ разреженных графовых данных. 