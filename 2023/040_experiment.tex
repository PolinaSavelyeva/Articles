% !TeX spellcheck = ru_RU
% !TEX root = vkr.tex

\section{Эксперимент}

В данном разделе будут рассмотрены результаты экспериментального исследования алгоритма, приведённого в предыдущих главах. Основная задача исследования: оценить производительность BFS в условиях, близких к реальным, сравнить полученные показатели и ответить на вопросы RQ1-RQ3.

\subsection{Условия эксперимента}

Эксперименты проводились на рабочей станции со следующими характеристиками:
\begin{itemize}
    \item Центральный процессор: AMD Ryzen 5 5600X 
    \item Количество ядер ЦПУ: 6-Core Processor, 12 Threads 
    \item Базовая тактовая частота ЦПУ: 3.70 GHz
    \item Объём оперативной памяти: 16.0 GB
    \item Операционная система: Windows 11 Pro, version 22H2
\end{itemize}

\subsection{Исследовательские вопросы}

\begin{itemize}
    \item[\textbf{RQ1:}] При каких параметрах графа выгоднее использовать параллельную версию алгоритма, а при каких последовательную?
    \item[\textbf{RQ2:}] Использование какого количества потоков даёт наибольший выигрыш в производительности?
    \item[\textbf{RQ3:}] Чем можно объяснить такое количество потоков?
\end{itemize}

\subsection{Метрики}

Для замеров производительности используется пакет BenchmarkDotNet v0.13.4. В качестве метрик производительности выступает время, требуемое для выполнения операции.

Для всех экспериментов используются неориентированные графы --- то есть графы с симметричной матрицей смежности. Примеры реальных данных взяты из MatrixМarket, разреженность этих данных не превышает 2 процента. Данные различной степени разреженности генерируются с помощью алгоритма листинг. какой-то. 

В данной работе результат работы БФС не зависит от типа весов на рёбрах (в ответе учитывается только порядок посещения вершин), то элементы матриц, взятых в качестве экспериментальных данных, будут принадлежать Pattern-типу. Другими словами, элемент либо есть (Some() в реализации) или его нет (None в реализации), без привязки к конкретному значению. Необходимость использования элементов одного строения для проведения серии экспериментов обусловлено тем, что сравнения и другие арифметические операции с одними типами могут быть более ресурсоёмкий по сравнению с другими.

\subsection{Результаты}

\subsubsection{RQ1} Пояснения
\subsubsection{RQ2} Пояснения
\subsubsection{RQ3} Пояснения
%% Пару слов о затратах
%% Пару слов я не придумала о чем